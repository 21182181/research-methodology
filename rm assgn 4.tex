\documentclass[10pt]{article}
\usepackage{apacite}
\usepackage{zed-csp,graphicx,color}%from
\begin{document}
\bibliographystyle{apacite}
\bibliography{References}
\begin{titlepage}
  \begin{figure}[h]
  \centerline{\small MAKERERE 
  \includegraphics[width=0.1\textwidth]{muk_log} UNIVERSITY}
\end{figure}
\centerline{COLLEGE OF COMPUTING AND INFORMATIC SCIENCES}
\paragraph{•}
\centerline{DEPARTMENT OF COMPUTER SCIENCE\\}
\paragraph{•}
\centerline{COURSEWORK: RESEARCH METHODOLOGY(BIT 2207)\\}
\paragraph{•}
\centerline{LECTURER: MR.ERNEST MWEBAZE}
\paragraph{•}
\centerline{TOPIC\\ Interactive media ads: Social media as a marketing tool.}  
\paragraph{•}
\centerline{COMPILED BY: \
 Nabwire Barbra}
 \paragraph{•}
\centerline{STUDENT NUMBER : 216011336 }
\paragraph{•}
\centerline{REGISTRATION NUMBER:16/U/8256/PS}
\paragraph{•}
\end{titlepage}
\pagenumbering{roman}
\newpage
\pagenumbering{arabic}
\section{Introduction}
In today’s technology driven world, social networking sites have become an avenue
where retailers can extend their marketing campaigns to a wider range of consumers. Chi (2011
46) defines social media marketing as a “connection between brands and consumers, [while]
offering a personal channel and currency for user centered networking and social interaction.”
The tools and approaches for communicating with customers have changed greatly with the
emergence of social media; therefore, businesses must learn how to use social media in a way
that is consistent with their business plan (Mangold and Faulds 2099). This is especially true for
companies striving to gain a competitive advantage.
\section{Method}
Viral advertising has become a way in which retailers are marketing and providing more
information on their brands or products. A viral approach to online advertising has a major
advantage because communication is more targeted to a brand’s intended consumer (Bampo et
al., 2008). This can be attributed to the fact that “viral communication affords the marketer a
greater degree of creative license through a message delivery medium that is more intimate and
personalized, thereby increasing the likelihood of reaching hard to get audience members”
(Bampo et al. 2008, 274). Viral advertising is “unpaid peer to peer communication of
provocative content originating from an identified sponsor using the Internet to persuade or
influence an audience to pass along the content to others” (Porter and Golan as cited by Chu
2011, 31). Viral advertising differs from UGC because an identified sponsor is associated with
the ad, thus signifying the origin of the ad and who created it. Numerous studies of viral
advertising have found that humor, sexuality, stealth, and positive experiences are relevant
factors that contribute to the success of viral advertising.
\section{conclusion}
Research has determined that retailers can increase awareness of their
brand by being creative when engaging customers on social media sites. “As more shoppers are
using social media (e.g., Twitter, Facebook, MySpace, and LinkedIn) and rely on them for
marketing shopping decisions, promotion through these media has become important” (Shankar
et al. 2011, 32). According to Curran et al. (2011), social media sites such as Facebook are better
than other advertising avenues because it stores information on all its users thus ensuring
marketing reaches a retailer’s specific target market.

  
\end{document}