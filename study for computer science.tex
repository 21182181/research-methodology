\documentclass[12pt,a4paper]{article}
\author{Nabwire Barbra Sandra}
\title{Why study computer science?}
\begin{document}

\maketitle
\textsf{Research methodology BIS 2207
\\
Lecturer: Mr. Ernest Mwebaze
\\Research topic: Why study computer science
\\Nabire Barbra Sandra
\\16/u/8256/ps}
\begin{center}
\textbf{
\section*{Introduction}}
\end{center}
\paragraph*{Computer science }
Computer science is the study of the theory, experimentation, and engineering that form the basis for the design and use of computers. It is the scientific and practical approach to computation and its applications and the systematic study of the feasibility, structure, expression, and mechanization of the methodical procedures (or algorithms) that underlie the acquisition, representation, processing, storage, communication of, and access to, information. An alternate, more succinct definition of computer science is the study of automating algorithmic processes that scale. A computer scientist specializes in the theory of computation and the design of computational systems.[1]
Its fields can be divided into a variety of theoretical and practical disciplines. Some fields, such as computational complexity theory (which explores the fundamental properties of computational and intractable problems), are highly abstract, while fields such as computer graphics emphasize real-world visual applications. Other fields still focus on challenges in implementing computation. For example, programming language theory considers various approaches to the description of computation, while the study of computer programming itself  considers the challenges in making computers and computations useful, usable, and universally accessible to humans

\begin{center}
\textbf{
\section*{Background}}
\end{center}
\paragraph*{History of computer science}
Computer Science is an exciting, growing, challenging field that has impact in most aspects of everyday life. These areas include medicine, communications, auotomotive technology, weather forcasting, entertainment, mining,  pharmacology, forensics, manufacturing, disaster recovery, security, law, business. For practically any area you can think of there is an application of computer technology. Yet there are still many new computer applications to be discovered and implemented in that area, and you could be involved in that exciting endeavor and service to humanity.
\begin{center}
\textbf{
\section*{Scope}}
\end{center}
Computer science graduates are some of the most sought-after graduates and earn among the highest salaries right out of college.  So while the news talks about how some of the routine jobs have gone “off shore”, there is still ample opportunity for talented computer science majors. 
The digital age needs computer scientists Like it or not you're living in it – this is the Digital Age. Computer programmes have all but infiltrated every aspect of our lives. Computer scientists theorise, design, develop, and apply the software and hardware for the programmes we use day in day out – sounds pretty important to us.
Computer scientists are needed in every type of industry Every industry uses computers so naturally computer scientists can work in any. Problems in science, engineering, health care, and so many other areas can be solved by computers. It's up to the computer scientist to figure out how, and design the software to apply the solution. Read our Computer Science Careers Guide.
Year abroad opportunities Computers have gone global, and it would be silly for Computer Science education providers to not reflect this fact. Check the opportunities for overseas study on the courses that interest you. A year abroad will provide you with a deeper understanding of how computers are used around the world, allowing you to experience other cultures, and gain some language skills in the process.

\begin{center}
\textbf{
\section*{Conclusion}}
\end{center}
Every industry uses computers so naturally computer scientists can work in any. Problems in science, engineering, health care, and so many other areas can be solved by computers. It's up to the computer scientist to figure out how, and design the software to apply the solution.
\end{document}